% Michael Black, Texas A&M University, Department of Agricultural Economics
% Template for a simple Beamer presentation using TAMU colors
%%%%%%%%%%%%%%%%%%%%%%%%%%%%%%%%%%%%%%%%%%%%%%%%%%%%%%%%%%%%%%%%%%%%%
%%%%%%%%%%%%%%%%%%%%%%%%%%%%%%%%%%%%%%%%%%%%%%%%%%%%%%%%%%%%%%%%%%%%%
\documentclass{beamer}
\usepackage{graphicx}
\usepackage{xcolor}
\usepackage{natbib}
\usepackage{hyperref}
\bibliographystyle{apalike}
\usetheme{boxes}
%Use the following theme for more color:
\usetheme{metropolis}
\usepackage{amsmath}

\definecolor{maroon}{RGB}{80,0,0}
\definecolor{tamwhite}{RGB}{255,255,255}
\definecolor{tamyellow}{RGB}{252,227,0}
\definecolor{tamred}{RGB}{228,0,43}
\definecolor{tamgrey}{RGB}{112,115,115}

\setbeamercolor{title}{fg = maroon}
\setbeamercolor{frametitle}{fg = tamwhite, bg = maroon}
\setbeamercolor{structure}{fg = tamgrey, bg = tamyellow}

\hypersetup{
    colorlinks=true,
    linkcolor=blue,
    filecolor=magenta,      
    urlcolor=cyan,
}

% Note, to include a LaTeX object (like output table), use:
% \input{file_name.tex}
% To include a graph or image, use:
% \includegraphics[scale=0.5]{file_name.png}

\title{Applied Production Economcis}
%\author{Michael Black\inst{1}}
%\institute[]{
 %   \inst{1}%
  %  Department of Agricultural Economics\\
   % Texas A\&M University
%}
\date{AGEC 317: Economic Analysis for Agribusiness Management \\ Instructor: Michael Black}

\titlegraphic{\begin{flushright} \vspace{6.5cm} \includegraphics[width=1.5cm]{agec.png} \end{flushright}}


%%%%%%%%%%%%%%%%%%%%%%%%%%%%%%%%%%%%%%%%%%%%%%%%%%%%%%%%%%%%%%%%%%%%%
%%%%%%%%%%%%%%%%%%%%%%%%%%%%%%%%%%%%%%%%%%%%%%%%%%%%%%%%%%%%%%%%%%%%%
\begin{document}
%%%%%%%%%%%%%%%%%%%%%%%%%%%%%%%%%%%%%%%%%%%%%%%%%%%%%%%%%%%%%%%%%%%%%
\begin{frame}[plain, noframenumbering]
  \titlepage
\end{frame}
%%%%%%%%%%%%%%%%%%%%%%%%%%%%%%%%%%%%%%%%%%%%%%%%%%%%%%%%%%%%%%%%%%%%%
\begin{frame}{Where are we?}
We can now build and interpret simple and complicated models. It is now time to apply this technical skill to real-world problems. \\
We start with the economics of \emph{production}.
\end{frame}
%%%%%%%%%%%%%%%%%%%%%%%%%%%%%%%%%%%%%%%%%%%%%%%%%%%%%%%%%%%%%%%%%%%%%
\begin{frame}{Production economics}
Quick review: 
\begin{itemize}
	\item \textbf{Production} is the process of using inputs to create a good or service that has economic value to consumers.
	\item A \textbf{production function} is the mathematical relationship that describes a firm's ability to transform inputs into output.
	\item The \textbf{short-run} time horizon is any time period where there are some fixed costs.
	\item The \textbf{long-run} time horizon is any period where all costs are variable.
\end{itemize}
\end{frame}
%%%%%%%%%%%%%%%%%%%%%%%%%%%%%%%%%%%%%%%%%%%%%%%%%%%%%%%%%%%%%%%%%%%%%
\begin{frame}{Production economics}
A \textbf{production function}:
$$q = f(x,y)$$
where $q$ is output, $x$ and $y$ are inputs.
$$MPP_x = \frac{\partial q}{\partial x}$$
$MPP_x$ is the \emph{marginal physical product of x}: it measures the change in output for a unit increase in the $x$ input.
\end{frame}
%%%%%%%%%%%%%%%%%%%%%%%%%%%%%%%%%%%%%%%%%%%%%%%%%%%%%%%%%%%%%%%%%%%%%
\begin{frame}{Production economics}
$$APP_x = \frac{q}{x}$$
$APP_x$ is the \emph{average physical product of x}: it is the ratio of total output of $q$ to amount of $x$ input.
\end{frame}
%%%%%%%%%%%%%%%%%%%%%%%%%%%%%%%%%%%%%%%%%%%%%%%%%%%%%%%%%%%%%%%%%%%%%
\begin{frame}{Production economics}
Stages of Production:
\begin{center}
	\includegraphics[scale=0.25]{stages.png}
\end{center}
\end{frame}
%%%%%%%%%%%%%%%%%%%%%%%%%%%%%%%%%%%%%%%%%%%%%%%%%%%%%%%%%%%%%%%%%%%%%
\begin{frame}{Production economics}
Stages of Production (for a single input):
\begin{itemize}
	\item Stage 1: MPP$>$APP
	\item Stage 2: APP$>$MPP \emph{and} MPP$>$0
	\item Stage 3: MPP $<$ 0
\end{itemize}
\end{frame}
%%%%%%%%%%%%%%%%%%%%%%%%%%%%%%%%%%%%%%%%%%%%%%%%%%%%%%%%%%%%%%%%%%%%%
\begin{frame}{Production economics}
Operating in Stage 1 of production is not desirable, because increasing input will still, on average, result in increased output. That's good. Stage 3 is not good because increasing output will actually \emph{decrease} output. Stage 2 is good. We want to operate somewhere in Stage 2.
\end{frame}
%%%%%%%%%%%%%%%%%%%%%%%%%%%%%%%%%%%%%%%%%%%%%%%%%%%%%%%%%%%%%%%%%%%%%
\begin{frame}{Production economics}
The \emph{elasticity of production} for an input $x$ is:
$$\varepsilon_x = \frac{\partial q}{\partial x}\cdot\frac{x}{y}=MPP_x \cdot \frac{1}{APP_x}=\frac{MPP_x}{APP_x}$$
\end{frame}
%%%%%%%%%%%%%%%%%%%%%%%%%%%%%%%%%%%%%%%%%%%%%%%%%%%%%%%%%%%%%%%%%%%%%
\begin{frame}{Production economics}
The \textbf{Law of Diminishing Returns} rules in production: eventually, you'll get less ``bang for the buck", and get to a point where adding inputs decreases output. Think about trying to use 4,000 lemons to create lemonade for a small lemonade stand. The 4,001st lemon isn't going to help you; it will just bury you in lemons and hurt production. 
\end{frame}
%%%%%%%%%%%%%%%%%%%%%%%%%%%%%%%%%%%%%%%%%%%%%%%%%%%%%%%%%%%%%%%%%%%%%
\begin{frame}{Production economics}
Suppose we have production with two inputs: $q = f(x,y)$. What is the optimal level of production? More precisely, how do we maximize output?
\begin{eqnarray*}
	q = f(x,y) \\
	\frac{\partial q}{\partial x} = \frac{\partial f(\cdot)}{\partial x} = MPP_x = 0 \\
	\frac{\partial q}{\partial y} = \frac{\partial f(\cdot)}{\partial y} = MPP_y = 0 \\
\end{eqnarray*}
Without constraints, to maximize production we would set the marginal physical product of both inputs to zero.
\end{frame}
%%%%%%%%%%%%%%%%%%%%%%%%%%%%%%%%%%%%%%%%%%%%%%%%%%%%%%%%%%%%%%%%%%%%%
\begin{frame}{Production economics}
Okay, but how do we know what the production function looks like? Everything depends on this. \\
\vspace{2cm}
It is up to us to estimate.
\end{frame}
%%%%%%%%%%%%%%%%%%%%%%%%%%%%%%%%%%%%%%%%%%%%%%%%%%%%%%%%%%%%%%%%%%%%%
\begin{frame}{Production economics}
If production typically looks like this:
\begin{center}
	\includegraphics[scale=0.25]{stages.png}
\end{center}
...what kind of model should we specify?
\end{frame}
%%%%%%%%%%%%%%%%%%%%%%%%%%%%%%%%%%%%%%%%%%%%%%%%%%%%%%%%%%%%%%%%%%%%%
\begin{frame}{Production economics}
\begin{itemize}
	\item A quadratic model allows us to estimate and upside down ``U-shaped" curve! Perfect!
	\item We can also use a ``log-log" model, like Cobb-Douglas. 
\end{itemize}
\end{frame}
%%%%%%%%%%%%%%%%%%%%%%%%%%%%%%%%%%%%%%%%%%%%%%%%%%%%%%%%%%%%%%%%%%%%%
\begin{frame}{Production economics}
Quadratic model:
$$q_i = \beta_0 + \beta_1capital_i + \beta_2capital_i^2 + \beta_3labor_i + \beta_4labor_i^2+u_i$$
\end{frame}
%%%%%%%%%%%%%%%%%%%%%%%%%%%%%%%%%%%%%%%%%%%%%%%%%%%%%%%%%%%%%%%%%%%%%
\begin{frame}{Production economics}
Max-production with quadratic model:
\begin{eqnarray*}
	q_i = \beta_0 + \beta_1capital_i + \beta_2capital_i^2 + \beta_3labor_i + \beta_4labor_i^2+u_i	\\
	\frac{\partial q}{\partial capital} = \beta_1 + 2\beta_2capital_i = 0 \\
	\frac{\partial q}{\partial labor} = \beta_3 + 2\beta_4labor_i = 0
\end{eqnarray*}
\begin{eqnarray*}
	\Rightarrow captial_i^* = \frac{-\beta_1}{2\beta_2} \\
	\Rightarrow labor_i^* = \frac{-\beta_3}{2\beta_4} 
\end{eqnarray*}

\end{frame}
%%%%%%%%%%%%%%%%%%%%%%%%%%%%%%%%%%%%%%%%%%%%%%%%%%%%%%%%%%%%%%%%%%%%%
\begin{frame}{Production economics}
Log-log model:
\begin{eqnarray*}
	q_i = AK^\alpha L^\beta \\
	\Rightarrow \ln(q_i) = \ln(A) + \alpha \ln(K) + \beta \ln(L) + u_i
\end{eqnarray*}
\end{frame}
%%%%%%%%%%%%%%%%%%%%%%%%%%%%%%%%%%%%%%%%%%%%%%%%%%%%%%%%%%%%%%%%%%%%%
\begin{frame}{Production economics}
Max-production with log-log model:
\begin{eqnarray*}
	\ln(q_i) = \ln(A) + \alpha \ln(K) + \beta \ln(L) + u_i \\
	q_i = AK^\alpha L^\beta \\
	\frac{\partial q}{\partial K} = \alpha AK^{1-\alpha}L^\beta = 0 \\
	\frac{\partial q}{\partial L} = \beta AK^{\alpha}L^{1-\beta} = 0 
\end{eqnarray*}
\end{frame}
%%%%%%%%%%%%%%%%%%%%%%%%%%%%%%%%%%%%%%%%%%%%%%%%%%%%%%%%%%%%%%%%%%%%%
\begin{frame}{Production economics}
The optimal solution now is a bit more complicated:
\begin{eqnarray*}
	\alpha AK^{1-\alpha}L^\beta = \beta AK^{\alpha}L^{1-\beta} \\
	 \Rightarrow K^* = \frac{\beta}{\alpha}L \\
	 \Rightarrow L^* = \frac{\alpha}{\beta}K 
\end{eqnarray*}
\end{frame}
%%%%%%%%%%%%%%%%%%%%%%%%%%%%%%%%%%%%%%%%%%%%%%%%%%%%%%%%%%%%%%%%%%%%%
\begin{frame}{Production economics}
Of course, both of these examples are without constraints. If we add constraints, then we have a constrained optimization problem where we need to use a Lagrangian to solve.
\end{frame}
%%%%%%%%%%%%%%%%%%%%%%%%%%%%%%%%%%%%%%%%%%%%%%%%%%%%%%%%%%%%%%%%%%%%%
\begin{frame}{Production economics}
To summarize, we:
\begin{enumerate}
	\item Observe data
	\item Use OLS to predict an appropriate model
	\item Use OLS regression results to form an optimization problem
	\item Solve the optimization problem
\end{enumerate}
\end{frame}
%%%%%%%%%%%%%%%%%%%%%%%%%%%%%%%%%%%%%%%%%%%%%%%%%%%%%%%%%%%%%%%%%%%%%
\begin{frame}{Costs and profit}
Okay, but our goal is seldom to simply maximize output. We may want to minimize costs or maximize profit.
\end{frame}
%%%%%%%%%%%%%%%%%%%%%%%%%%%%%%%%%%%%%%%%%%%%%%%%%%%%%%%%%%%%%%%%%%%%%
\begin{frame}{Costs and profit}
\begin{itemize}
	\item \textbf{Total value product} of inputs: $TVP = P\cdot f(x,y)$, where $P$ is the price of the sold output.
	\item \textbf{Marginal value product} of input $x$: $MVP_x = P\cdot MPP_x$, where $P$ is the price of the sold output.
	\item \textbf{Average value product} of input $x$: $AVP_x = P\cdot APP_x$, where $P$ is the price of the sold output.
	\item \textbf{Total cost} function: $c = r_xx + r_yy+b$, where $r_x$ is the per-unit cost of input $x$ (same for $y$), and $b$ is some fixed cost.
\end{itemize}
\end{frame}
%%%%%%%%%%%%%%%%%%%%%%%%%%%%%%%%%%%%%%%%%%%%%%%%%%%%%%%%%%%%%%%%%%%%%
\begin{frame}{Costs and profit}
\begin{itemize}
	\item \textbf{Profit}: $\pi = 	P\cdot q  - r_xx - r_yy-b$, where $q$ is the estimated production function. 
\end{itemize}
\end{frame}
%%%%%%%%%%%%%%%%%%%%%%%%%%%%%%%%%%%%%%%%%%%%%%%%%%%%%%%%%%%%%%%%%%%%%
\begin{frame}{Costs and profit}
Profit function with quadratic form:
$$\pi = 	P\cdot (\color{red} \beta_0 + \beta_1K_i + \beta_2K_i^2 + \beta_3 L_i + \beta_4 L_i^2\color{black})  - r_KK - r_LL-b$$
Where the \color{red} red \color{black} portion comes from an estimated econometric model.
\end{frame}
%%%%%%%%%%%%%%%%%%%%%%%%%%%%%%%%%%%%%%%%%%%%%%%%%%%%%%%%%%%%%%%%%%%%%
\begin{frame}{Example}
\begin{center}
	\includegraphics[scale = 0.3]{tp.png}
\end{center}
Consider the production process for toilet paper...
\end{frame}
%%%%%%%%%%%%%%%%%%%%%%%%%%%%%%%%%%%%%%%%%%%%%%%%%%%%%%%%%%%%%%%%%%%%%
\begin{frame}{Example}
Inputs:
\begin{itemize}
	\item Human labor
	\item Electricity to run machines
	\item Recycled paper
	\item Glue
	\item Bleach
\end{itemize}
\end{frame}
%%%%%%%%%%%%%%%%%%%%%%%%%%%%%%%%%%%%%%%%%%%%%%%%%%%%%%%%%%%%%%%%%%%%%
\begin{frame}{Example}
\begin{eqnarray*}
	\pi = P\cdot (\beta_0 + \beta_1electricity_i + \beta_2electricity_i^2 + \beta_3 paper_i + \beta_4 paper_i^2) \\
	- r_{paper}paper - r_{electricity}electricity - b
\end{eqnarray*}
\end{frame}
%%%%%%%%%%%%%%%%%%%%%%%%%%%%%%%%%%%%%%%%%%%%%%%%%%%%%%%%%%%%%%%%%%%%%
\begin{frame}{Example}
Let's choose the optimal amount of electricity and recycled paper to use to maximize profit, \emph{without any constraints}:
\begin{eqnarray*}
	\frac{\partial \pi}{\partial electricity} = P(\beta_1 + 2\beta_2)-r_{electricity} = 0 \\
	\Rightarrow electricity^* = \frac{\frac{r_{electricity}}{P}-\beta_1}{2\beta_2}
\end{eqnarray*}
\end{frame}
%%%%%%%%%%%%%%%%%%%%%%%%%%%%%%%%%%%%%%%%%%%%%%%%%%%%%%%%%%%%%%%%%%%%%
\begin{frame}{Example}
...we can find the same optimal input use for recycled paper as well. Now what if we have a budget constraint? 
\end{frame}
%%%%%%%%%%%%%%%%%%%%%%%%%%%%%%%%%%%%%%%%%%%%%%%%%%%%%%%%%%%%%%%%%%%%%
\begin{frame}{Example}
Now let's solve a real-world production problem.
\end{frame}
%%%%%%%%%%%%%%%%%%%%%%%%%%%%%%%%%%%%%%%%%%%%%%%%%%%%%%%%%%%%%%%%%%%%%
\begin{frame}{Problem}
Suppose you work for a large international agribusiness corporation: \emph{Lentils-n-More} owns a large lentil operation in India. Your firm has run an experiment on some of its fields. Each field receives a randomly selected amount of nitrogen fertilizer, and a randomly selected amount of seed sachets to plant in the field. Some farms are allowed to operate totally independently, while others must report the entire growing process to \emph{Lentils-n-More}. 
\end{frame}
%%%%%%%%%%%%%%%%%%%%%%%%%%%%%%%%%%%%%%%%%%%%%%%%%%%%%%%%%%%%%%%%%%%%%
\begin{frame}{Problem}
Your goal is to determine the profit-maximizing level of nitrogen fertilizer and seed sachets to purchase for your India operation. \\
Your secondary goal is to confirm that your India operation is in Stage II production, and not Stage I or III.
\end{frame}
%%%%%%%%%%%%%%%%%%%%%%%%%%%%%%%%%%%%%%%%%%%%%%%%%%%%%%%%%%%%%%%%%%%%%
\begin{frame}{Data Description}
\begin{itemize}
	\item Yield: yield of lentils, in bushels per ha
	\item Nitrogen: amount of nitrogen fertilizer concentrate, in gallons
	\item SeedPop: the seed population, measured in number of seed sachets available locally
	\item Ownership: =1 if operation is monitored by your firm, =0 if no oversight
\end{itemize}
\end{frame}
%%%%%%%%%%%%%%%%%%%%%%%%%%%%%%%%%%%%%%%%%%%%%%%%%%%%%%%%%%%%%%%%%%%%%
\begin{frame}{Solution: Step-by-step}
\begin{itemize}
	\item Estimate model that is quadratic in both inputs, plus interaction, plus dummy variable
	\item Use results and current prices to predict profit
	\item Solve for optimal by hand and substitute parameters, or use Solver in Excel
\end{itemize}
\end{frame}
%%%%%%%%%%%%%%%%%%%%%%%%%%%%%%%%%%%%%%%%%%%%%%%%%%%%%%%%%%%%%%%%%%%%%
%%%%%%%%%%%%%%%%%%%%%%%%%%%%%%%%%%%%%%%%%%%%%%%%%%%%%%%%%%%%%%%%%%%%%
\end{document}