% Michael Black, Texas A&M University, Department of Agricultural Economics
% Template for a simple Beamer presentation using TAMU colors
%%%%%%%%%%%%%%%%%%%%%%%%%%%%%%%%%%%%%%%%%%%%%%%%%%%%%%%%%%%%%%%%%%%%%
%%%%%%%%%%%%%%%%%%%%%%%%%%%%%%%%%%%%%%%%%%%%%%%%%%%%%%%%%%%%%%%%%%%%%
\documentclass{beamer}
\usepackage{graphicx}
\usepackage{xcolor}
\usepackage{natbib}
\usepackage{hyperref}
\bibliographystyle{apalike}
\usetheme{boxes}
%Use the following theme for more color:
\usetheme{metropolis}
\usepackage{amsmath}

\definecolor{maroon}{RGB}{80,0,0}
\definecolor{tamwhite}{RGB}{255,255,255}
\definecolor{tamyellow}{RGB}{252,227,0}
\definecolor{tamred}{RGB}{228,0,43}
\definecolor{tamgrey}{RGB}{112,115,115}

\setbeamercolor{title}{fg = maroon}
\setbeamercolor{frametitle}{fg = tamwhite, bg = maroon}
\setbeamercolor{structure}{fg = tamgrey, bg = tamyellow}

\hypersetup{
    colorlinks=true,
    linkcolor=blue,
    filecolor=magenta,      
    urlcolor=cyan,
}

% Note, to include a LaTeX object (like output table), use:
% \input{file_name.tex}
% To include a graph or image, use:
% \includegraphics[scale=0.5]{file_name.png}

\title{Advanced Model Specifications}
%\author{Michael Black\inst{1}}
%\institute[]{
 %   \inst{1}%
  %  Department of Agricultural Economics\\
   % Texas A\&M University
%}
\date{AGEC 317: Economic Analysis for Agribusiness Management \\ Instructor: Michael Black}

\titlegraphic{\begin{flushright} \vspace{6.5cm} \includegraphics[width=1.5cm]{agec.png} \end{flushright}}


%%%%%%%%%%%%%%%%%%%%%%%%%%%%%%%%%%%%%%%%%%%%%%%%%%%%%%%%%%%%%%%%%%%%%
%%%%%%%%%%%%%%%%%%%%%%%%%%%%%%%%%%%%%%%%%%%%%%%%%%%%%%%%%%%%%%%%%%%%%
\begin{document}
%%%%%%%%%%%%%%%%%%%%%%%%%%%%%%%%%%%%%%%%%%%%%%%%%%%%%%%%%%%%%%%%%%%%%
\begin{frame}[plain, noframenumbering]
  \titlepage
\end{frame}
%%%%%%%%%%%%%%%%%%%%%%%%%%%%%%%%%%%%%%%%%%%%%%%%%%%%%%%%%%%%%%%%%%%%%
\begin{frame}{Where are we?}
We can run univariate linear regressions:
$$quantity_i = \beta_0 + \beta_1 price_i + \varepsilon_i$$
We can run multivariate linear regressions:
$$quantity_i = \beta_0 + \beta_1 price_i + \beta_2 income_i + \varepsilon_i$$
...and we can identify when a model might be ``bad".
\end{frame}
%%%%%%%%%%%%%%%%%%%%%%%%%%%%%%%%%%%%%%%%%%%%%%%%%%%%%%%%%%%%%%%%%%%%%
\begin{frame}{Where are we?}
But does our model always have to produce straight lines? \\
No, and that is the point of this lecture. We'll discuss 4 ways of making models with different kinds of lines:
\begin{itemize}
	\item ``Log" models
	\item Quadratic models
	\item Dummy variable models
	\item Interaction models
\end{itemize}
\end{frame}
%%%%%%%%%%%%%%%%%%%%%%%%%%%%%%%%%%%%%%%%%%%%%%%%%%%%%%%%%%%%%%%%%%%%%
\begin{frame}{Log models}
$$y=\ln(x)$$
\begin{center}
	\includegraphics[scale = 0.5]{log.pdf}
\end{center}
Taking the ``log'' of x will make the y-values corresponding to large x-values much smaller. However, the log of zero does not exist. If we have zeros in our data, \textbf{we cannot take the log of that variable}.
\end{frame}
%%%%%%%%%%%%%%%%%%%%%%%%%%%%%%%%%%%%%%%%%%%%%%%%%%%%%%%%%%%%%%%%%%%%%
\begin{frame}{Log models}
Consider the ``Soda\_Exp.xlsx" data in eCampus. Suppose we estimate:
$$Soda\quad Expenditures_i = \beta_0 + \beta_1 Income_i + \varepsilon_i$$
\end{frame}
%%%%%%%%%%%%%%%%%%%%%%%%%%%%%%%%%%%%%%%%%%%%%%%%%%%%%%%%%%%%%%%%%%%%%
\begin{frame}{Log models}
We are predicting that as you earn an additional dollar of income, you will spend \$0.0016 more on soda in a given year. So if you find a dollar on the ground, \%15 of a single penny of that dollar will go to soda throughout the year.
\end{frame}
%%%%%%%%%%%%%%%%%%%%%%%%%%%%%%%%%%%%%%%%%%%%%%%%%%%%%%%%%%%%%%%%%%%%%
\begin{frame}{Log models}
Come on! That's not intuitive at all. An option would be to take the log of income, then perform the same OLS regression. In fact, lets perform the following regressions:
\begin{eqnarray}
	Expenditures_i &=& \beta_0 + \beta_1 Income_i + \varepsilon_i \\
	Expenditures_i &=& \beta_0 + \beta_1 \ln(Income_i) + \varepsilon_i \\
	\ln(Expenditures_i) &=& \beta_0 + \beta_1 Income_i + \varepsilon_i \\
	\ln(Expenditures_i) &=& \beta_0 + \beta_1 \ln(Income_i) + \varepsilon_i 
\end{eqnarray}
(Remember that we need to filter out the zeros before taking the log of a variable)
\end{frame}
%%%%%%%%%%%%%%%%%%%%%%%%%%%%%%%%%%%%%%%%%%%%%%%%%%%%%%%%%%%%%%%%%%%%%
\begin{frame}{Log models}
Before interpreting the models, take a look at the $R^2$. Which model performs best?
\end{frame}
%%%%%%%%%%%%%%%%%%%%%%%%%%%%%%%%%%%%%%%%%%%%%%%%%%%%%%%%%%%%%%%%%%%%%
\begin{frame}{Log models}
Once you perform a log-transformation on \emph{either} the LHS or RHS of a regression equation, the interpretation of the coefficient changes. Specifically, if we have the following general interpretations:
$$y = \beta_0 + \beta_1 x + \varepsilon$$
\begin{center}
\resizebox{\linewidth}{!}{
\begin{tabular}{llll}
Model              & Dependent Variable & Independent Variable & Interpretation of $\beta_1$ \\ \hline
Linear - Linear & y & x& $\beta_1 = \frac{\Delta y}{\Delta x}$\\
 & & & \\
Linear - Log & y  & ln(x) & $\beta_1 = \frac{\Delta y}{\% \Delta x}\cdot 100$ \\  
 & & & \\    
Log - Linear & ln(y)  & x & $\beta_1 = \frac{\% \Delta y}{\Delta x \cdot 100}$ \\
 & & & \\
Log - Log & ln(y)  & ln(x) & $\beta_1 = \frac{\% \Delta y}{\% \Delta x}$ \\ \hline
\end{tabular}}
\end{center}
\end{frame}
%%%%%%%%%%%%%%%%%%%%%%%%%%%%%%%%%%%%%%%%%%%%%%%%%%%%%%%%%%%%%%%%%%%%%
\begin{frame}{Log models}
In words:
\begin{center}
\resizebox{\linewidth}{!}{
\begin{tabular}{llll}
Model              & Dependent Variable & Independent Variable & Interpretation of $\beta_1$ \\ \hline
Linear - Linear & y & x& A one unit increase in x results in a $\beta_1$ unit increase/decrease in y\\
 & & & \\
Linear - Log & y  & ln(x) & A one percent increase in x results in a $\frac{\beta_1}{100}$ unit increase/decrease in y \\  
 & & & \\    
Log - Linear & ln(y)  & x & A one unit increase in x results in a $\beta_1 \times 100$ percent increase/decrease in y \\
 & & & \\
Log - Log & ln(y)  & ln(x) & A one percent increase in x results in a $\beta_1$ percent increase/decrease in y \\ \hline
\end{tabular}}
\end{center}

\end{frame}
%%%%%%%%%%%%%%%%%%%%%%%%%%%%%%%%%%%%%%%%%%%%%%%%%%%%%%%%%%%%%%%%%%%%%
\begin{frame}{Log models}
Interpreting the results of the models:
\begin{itemize}
	\item $Expenditures_i = \beta_0 + \beta_1 Income_i + \varepsilon_i$: A dollar increase in income results in a \$0.0016 increase in soda expenditures.
	\item $Expenditures_i = \beta_0 + \beta_1 \ln(Income_i) + \varepsilon_i$: A 1\% increase in income results in a \$0.70 increase in soda expenditures.
	\item $\ln(Expenditures_i) = \beta_0 + \beta_1 Income_i + \varepsilon_i$: A dollar increase in income results in a 0.00007\% increase in soda expenditures.
	\item $\ln(Expenditures_i) = \beta_0 + \beta_1 \ln(Income_i) + \varepsilon_i$: A 1\% increase in income results in a 0.26\% increase in soda expenditures.\footnote{Actually, the effect is zero, but I assume significance just for the example.}
\end{itemize}
\end{frame}
%%%%%%%%%%%%%%%%%%%%%%%%%%%%%%%%%%%%%%%%%%%%%%%%%%%%%%%%%%%%%%%%%%%%%
\begin{frame}{Log models}
Taking the log of some or all of our variables and \emph{then} performing OLS is totally fine, but we have to be very careful about the interpretations of the coefficients.
\end{frame}
%%%%%%%%%%%%%%%%%%%%%%%%%%%%%%%%%%%%%%%%%%%%%%%%%%%%%%%%%%%%%%%%%%%%%
\begin{frame}{Log models}
Another important example:
$$Y=AL^\alpha K^\beta e^\varepsilon$$
...the Cobb-Douglas production function!
\end{frame}
%%%%%%%%%%%%%%%%%%%%%%%%%%%%%%%%%%%%%%%%%%%%%%%%%%%%%%%%%%%%%%%%%%%%%
\begin{frame}{Log models}
The log of the Cobb-Douglas function:
$$\ln(Y) = \ln(A) + \alpha\ln(L) + \beta\ln(K) + \varepsilon$$
Suddenly, we have an estimable model! More on this in this module's problem set.
\end{frame}
%%%%%%%%%%%%%%%%%%%%%%%%%%%%%%%%%%%%%%%%%%%%%%%%%%%%%%%%%%%%%%%%%%%%%
\begin{frame}{Quadratic models}
A \textbf{quadratic model} allows for the classic ``U-shaped'' line:
\begin{center}
	\includegraphics[scale=0.5]{quad.pdf}
\end{center}
\end{frame}
%%%%%%%%%%%%%%%%%%%%%%%%%%%%%%%%%%%%%%%%%%%%%%%%%%%%%%%%%%%%%%%%%%%%%
\begin{frame}{Quadratic models}
...or the upside-down U-shaped line:
\begin{center}
	\includegraphics[scale=0.5]{negquad.pdf}
\end{center}
\end{frame}
%%%%%%%%%%%%%%%%%%%%%%%%%%%%%%%%%%%%%%%%%%%%%%%%%%%%%%%%%%%%%%%%%%%%%
\begin{frame}{Quadratic models}
A simple univariate quadratic model is:
$$y = \beta_0 + \beta_1 x + \beta_2 x^2 + \varepsilon$$
where $y$ is a function of x, but now we have two $x$-terms!
\end{frame}
%%%%%%%%%%%%%%%%%%%%%%%%%%%%%%%%%%%%%%%%%%%%%%%%%%%%%%%%%%%%%%%%%%%%%
\begin{frame}{Quadratic models}
So if our model is:
$$y = \beta_0 + \beta_1 x + \beta_2 x^2 + \varepsilon$$
What is the partial effect of $x$ on $y$?
$$ \Rightarrow \frac{\partial y}{\partial x} = $$
\end{frame}
%%%%%%%%%%%%%%%%%%%%%%%%%%%%%%%%%%%%%%%%%%%%%%%%%%%%%%%%%%%%%%%%%%%%%
\begin{frame}{Quadratic models}
$$ \Rightarrow \frac{\partial y}{\partial x} = \beta_1 + 2\beta_2x$$
So the slope is a \emph{function} of $x$. That is, the effect of $x$ on $y$ (slope) is different depending on what value of $x$ we are talking about. 
\end{frame}
%%%%%%%%%%%%%%%%%%%%%%%%%%%%%%%%%%%%%%%%%%%%%%%%%%%%%%%%%%%%%%%%%%%%%
\begin{frame}{Quadratic models}
What if we want to make a model that predicts the CPI given any year?
\begin{center}
	\includegraphics[scale=0.4]{cpi.png}
\end{center}
\end{frame}
%%%%%%%%%%%%%%%%%%%%%%%%%%%%%%%%%%%%%%%%%%%%%%%%%%%%%%%%%%%%%%%%%%%%%
\begin{frame}{Quadratic models}
Excel aside: open the ``CPI.xlsx" document and let's see how we can draw models directly on scatterplots.
\end{frame}
%%%%%%%%%%%%%%%%%%%%%%%%%%%%%%%%%%%%%%%%%%%%%%%%%%%%%%%%%%%%%%%%%%%%%
\begin{frame}{Quadratic models}
Using ``CPI.xlsx", estimate the following model:
$$CPI_t = \beta_0 + \beta_1 year_t + \beta_2 year_t^2+ \varepsilon_t$$
Important: this is time series data, which we will return to later. There are a couple technical assumptions we are ignoring here.
\end{frame}
%%%%%%%%%%%%%%%%%%%%%%%%%%%%%%%%%%%%%%%%%%%%%%%%%%%%%%%%%%%%%%%%%%%%%
\begin{frame}{Quadratic models}
Result:
$$CPI_t = 137,877.50 - (142.52)year_t + (0.04) year_t^2$$
Predicted CPI in 2020:
$$CPI_{2020} = 137,877.50 - (142.52)2020 + (0.04)2020^2 = 276.76$$
\end{frame}
%%%%%%%%%%%%%%%%%%%%%%%%%%%%%%%%%%%%%%%%%%%%%%%%%%%%%%%%%%%%%%%%%%%%%
\begin{frame}
What is the partial effect of year on CPI? What is, what is the anticipated change in CPI as a year progresses?
$$CPI_t = \beta_0 + \beta_1 year_t + \beta_2 year_t^2+ \varepsilon_t$$
$$\frac{\partial CPI}{\partial year} = \beta_1 + 2 \beta_2 year$$
Partial effect of time on CPI in 1930:
$$\frac{\partial CPI}{\partial year} = -142.52 + 2 (0.04) (1930) = -0.35$$
Partial effect of time on CPI in 2019:
$$\frac{\partial CPI}{\partial year} = -142.52 + 2 (0.04) (2019) = 6.21$$
\end{frame}
%%%%%%%%%%%%%%%%%%%%%%%%%%%%%%%%%%%%%%%%%%%%%%%%%%%%%%%%%%%%%%%%%%%%%
\begin{frame}{Quadratic models}
Another great aspect of quadratic models: they have clear maximums and minimums! We can ask, for example, in what year did CPI reach its minimum:
$$CPI_t = \beta_0 + \beta_1 year_t + \beta_2 year_t^2+ \varepsilon_t$$
$$\frac{\partial CPI}{\partial year} = \beta_1 + 2 \beta_2 year = 0$$
$$\Rightarrow year^* = -\frac{\beta_1}{2\beta_2} = \frac{142.52}{2\times 0.04} = 1934$$
\end{frame}
%%%%%%%%%%%%%%%%%%%%%%%%%%%%%%%%%%%%%%%%%%%%%%%%%%%%%%%%%%%%%%%%%%%%%
\begin{frame}{Interaction terms}
Consider the following demand function for coffee:
$$Q_i = \beta_0 + \beta_1 P_i + \beta_2 Income_i + \beta_3 HouseholdSize_i + \varepsilon_i$$
[Endogeneity alert!!!]
\end{frame}
%%%%%%%%%%%%%%%%%%%%%%%%%%%%%%%%%%%%%%%%%%%%%%%%%%%%%%%%%%%%%%%%%%%%%
\begin{frame}{Interaction terms}
What if the effect of income on coffee demand changed based on how large a household was? That is, the effect of a promotion for a household has a \emph{different} effect for small households than for large households. How can we make a model that allows for that? \\
Answer: interactions terms!
\end{frame}
%%%%%%%%%%%%%%%%%%%%%%%%%%%%%%%%%%%%%%%%%%%%%%%%%%%%%%%%%%%%%%%%%%%%%
\begin{frame}{Interaction terms}
\begin{eqnarray*}
	Q_i = \beta_0 + \beta_1 P_i + \beta_2 Income_i + \beta_3 HouseholdSize_i \\
	+ \beta_4 (Income_i \times HouseholdSize_i ) + \varepsilon_i
\end{eqnarray*}
\begin{eqnarray*}
	\frac{\partial Q_i }{\partial Income_i}= \beta_2 + \beta_4 HouseholdSize_i
\end{eqnarray*}
The marginal effect of income is now a function of household size.
\end{frame}
%%%%%%%%%%%%%%%%%%%%%%%%%%%%%%%%%%%%%%%%%%%%%%%%%%%%%%%%%%%%%%%%%%%%%
\begin{frame}{Interaction terms}
Open ``Coffee.xlsx". Estimate the following model:
\begin{eqnarray*}
	Q_i = \beta_0 + \beta_1 P_i + \beta_2 Income_i + \beta_3 HouseholdSize_i \\
	+ \beta_4 (Income_i \times HouseholdSize_i ) + \varepsilon_i
\end{eqnarray*}
Does income affect demand? Does the partial effect of income change with different household sizes?
\end{frame}
%%%%%%%%%%%%%%%%%%%%%%%%%%%%%%%%%%%%%%%%%%%%%%%%%%%%%%%%%%%%%%%%%%%%%
\begin{frame}{Dummy variables}
Our variables have been mostly \textbf{quantitative}. That is, they have meaningful numerical values. What if we wanted to include a variable that has no inherent value? \\
That is, a \textbf{qualitative} variable.
\end{frame}
%%%%%%%%%%%%%%%%%%%%%%%%%%%%%%%%%%%%%%%%%%%%%%%%%%%%%%%%%%%%%%%%%%%%%
\begin{frame}{Dummy variables}
Examples of dummy variables:
\begin{itemize}
	\item Political party affiliation
	\item Hair color
	\item Region
	\item ``Did you like the Star Wars prequels?''
	\item Many, many more
\end{itemize}
\end{frame}
%%%%%%%%%%%%%%%%%%%%%%%%%%%%%%%%%%%%%%%%%%%%%%%%%%%%%%%%%%%%%%%%%%%%%
\begin{frame}{Dummy variables}
Dummy variables add new lines to the regression model. Consider the following model:
$$Expenditures_i = \beta_0 + \beta_1 Income_i + \beta_2 Gender_i + \varepsilon_i$$

$\Rightarrow$ Dummy variables equal 1 or 0. In this case, $Gender_i=1$ if the person $i$ is female, and 0 otherwise. 
\end{frame}
%%%%%%%%%%%%%%%%%%%%%%%%%%%%%%%%%%%%%%%%%%%%%%%%%%%%%%%%%%%%%%%%%%%%%
\begin{frame}{Dummy variable regression}
Original model:
$$Expenditures_i = \beta_0 + \beta_1 Income_i + \beta_2 Gender_i + \varepsilon_i$$
Model for females:
\begin{eqnarray*}
	Expenditures_i = \beta_0 + \beta_1 Income_i + \beta_2 (1) + \varepsilon_i \\
	=(\beta_0 + \beta_2) +\beta_1 Income_i  + \varepsilon_i
\end{eqnarray*}
\end{frame}
%%%%%%%%%%%%%%%%%%%%%%%%%%%%%%%%%%%%%%%%%%%%%%%%%%%%%%%%%%%%%%%%%%%%%
\begin{frame}{Dummy variable regression}
Model for males:
\begin{eqnarray*}
	Expenditures_i = \beta_0 + \beta_1 Income_i + \beta_2 (0) + \varepsilon_i \\
	=\beta_0  +\beta_1 Income_i  + \varepsilon_i
\end{eqnarray*}
\end{frame}
%%%%%%%%%%%%%%%%%%%%%%%%%%%%%%%%%%%%%%%%%%%%%%%%%%%%%%%%%%%%%%%%%%%%%
\begin{frame}{Dummy variable regression}
The partial effect of income is the same for now, but males and females are allowed to have a different intercept, and thus a different ``wedge" between them. \\
Open ``SodaExp.xlsx". Estimate the following model:
$$Expenditures_i = \beta_0 + \beta_1 Income_i + \beta_2 Gender_i + \varepsilon_i$$
Is there a significant difference in the amount of soda expenditures between men and women?
\end{frame}
%%%%%%%%%%%%%%%%%%%%%%%%%%%%%%%%%%%%%%%%%%%%%%%%%%%%%%%%%%%%%%%%%%%%%
\begin{frame}{Dummy variable regression}
Important: if your qualitative variable has $n$ categories, you need to include $n-1$ dummy variables. Including a dummy variable for each category results in multi-colinearity (an OLS violation!)
Run the following model to see why:
$$Expenditures_i = \beta_0 + \beta_1 Income_i + \beta_2 E1_i + \beta_3 E2_i + \beta_4 E3_i + \beta_5 E4_i+  \varepsilon_i$$
\end{frame}
%%%%%%%%%%%%%%%%%%%%%%%%%%%%%%%%%%%%%%%%%%%%%%%%%%%%%%%%%%%%%%%%%%%%%
\begin{frame}{Dummy variable regression}
You need to omit one category, and this is called the \textbf{base category}. The interpretations of the variables are in relation to the base category.
Run the following model and interpret the results:
$$Expenditures_i = \beta_0 + \beta_1 Income_i + \beta_2 E1_i + \beta_3 E2_i + \beta_4 E3_i +  \varepsilon_i$$
Note that E4 is omitted and it thus the base category.
\end{frame}
%%%%%%%%%%%%%%%%%%%%%%%%%%%%%%%%%%%%%%%%%%%%%%%%%%%%%%%%%%%%%%%%%%%%%
\begin{frame}{Dummy variable regression}
\begin{itemize}
	\item $\beta_2 = 0 \Rightarrow$ people with less than a high school degree spend the same on soda as college graduates
	\item $\beta_3 = 0 \Rightarrow$ people with a high school degree spend the same on soda as college graduates
	\item $\beta_4 = 0 \Rightarrow$ people with some college experience spend the same on soda as college graduates
\end{itemize}
If $\beta_4$ were significant, then we would say that people with some college experience spend, on average, \$48 less than college graduates on soda. 
\end{frame}
%%%%%%%%%%%%%%%%%%%%%%%%%%%%%%%%%%%%%%%%%%%%%%%%%%%%%%%%%%%%%%%%%%%%%
\begin{frame}{Dummy variable regression}
Dummy variable regression is used frequently. We often want to model the shift of an intercept. That's the same as a shift in an entire curve! So dummy variable regression can capture:
\begin{itemize}
	\item Shifts in demand curve
	\item Shifts in supply curve
	\item Shift in cost curve from a per-unit tax
	\item etc
\end{itemize}
\end{frame}
%%%%%%%%%%%%%%%%%%%%%%%%%%%%%%%%%%%%%%%%%%%%%%%%%%%%%%%%%%%%%%%%%%%%%
\begin{frame}{Dummy variables with interaction terms}
We can also make models with dummy variables interacted with other terms. Then we can have different intercepts for different categories, \emph{and} allow the different categories to have different slopes. Suppose we have the following model:
$$wage_i = \beta_0 + \beta_1 tenure_i + \beta_2 female_i + \beta_3 (tenure_i \times female_i) + \varepsilon_i$$
With this model, ignoring endogeneity, we can answer the following questions:
\begin{itemize}
	\item Do women earn less on average than men?
	\item Does on-the-job experience matter less for women than men? (In terms of getting a raise)
\end{itemize}
\end{frame}
%%%%%%%%%%%%%%%%%%%%%%%%%%%%%%%%%%%%%%%%%%%%%%%%%%%%%%%%%%%%%%%%%%%%%
\begin{frame}{Dummy variables with interaction terms}
Open ``wage.xlsx" and estimate the following model:
$$wage_i = \beta_0 + \beta_1 tenure_i + \beta_2 female_i + \beta_3 (tenure_i \times female_i) + \varepsilon_i$$
\end{frame}
%%%%%%%%%%%%%%%%%%%%%%%%%%%%%%%%%%%%%%%%%%%%%%%%%%%%%%%%%%%%%%%%%%%%%
\begin{frame}{Dummy variables with interaction terms}
To interpret the effects, think of what the model is like for men vs. women. \\
Model for women:
\begin{eqnarray*}
	wage_i = \beta_0 + \beta_1 tenure_i + \beta_2(1) + \beta_3 (tenure_i \times 1) + \varepsilon_i \\
	= (\beta_2 +\beta_0) + (\beta_1 + \beta_3) tenure_i + \varepsilon_i
\end{eqnarray*}
Model for men:
\begin{eqnarray*}
	wage_i = \beta_0 + \beta_1 tenure_i + \beta_2(0) + \beta_3 (tenure_i \times 0) + \varepsilon_i \\
	= \beta_0 + (\beta_1) tenure_i + \varepsilon_i
\end{eqnarray*}
\end{frame}
%%%%%%%%%%%%%%%%%%%%%%%%%%%%%%%%%%%%%%%%%%%%%%%%%%%%%%%%%%%%%%%%%%%%%
\begin{frame}{Dummy variables with interaction terms}
With our data and model, women earn \$1.58 ($\beta_2$) less than men in the same profession, and while men earn a raise of \$0.18 ($\beta_1$) per year, women earn only a \$0.07 ($\beta_1 + \beta_3$) per year. 
\end{frame}
%%%%%%%%%%%%%%%%%%%%%%%%%%%%%%%%%%%%%%%%%%%%%%%%%%%%%%%%%%%%%%%%%%%%%
\begin{frame}{Key takeaways}
After finishing this module, you should be able to estimate and interpret the following types of models:
\begin{enumerate}
	\item Log models
	\item Quadratic models
	\item Dummy variable models
	\item Interaction term models
	\item Dummy variable with interaction term models
\end{enumerate}
\end{frame}
%%%%%%%%%%%%%%%%%%%%%%%%%%%%%%%%%%%%%%%%%%%%%%%%%%%%%%%%%%%%%%%%%%%%%
%%%%%%%%%%%%%%%%%%%%%%%%%%%%%%%%%%%%%%%%%%%%%%%%%%%%%%%%%%%%%%%%%%%%%
\end{document}