% Michael Black, Texas A&M University, Department of Agricultural Economics
% Template for a simple Beamer presentation using TAMU colors
%%%%%%%%%%%%%%%%%%%%%%%%%%%%%%%%%%%%%%%%%%%%%%%%%%%%%%%%%%%%%%%%%%%%%
%%%%%%%%%%%%%%%%%%%%%%%%%%%%%%%%%%%%%%%%%%%%%%%%%%%%%%%%%%%%%%%%%%%%%
\documentclass{beamer}
\usepackage{graphicx}
\usepackage{xcolor}
\usepackage{natbib}
\usepackage{hyperref}
\bibliographystyle{apalike}
\usetheme{boxes}
%Use the following theme for more color:
\usetheme{metropolis}
\usepackage{amsmath}

\definecolor{maroon}{RGB}{80,0,0}
\definecolor{tamwhite}{RGB}{255,255,255}
\definecolor{tamyellow}{RGB}{252,227,0}
\definecolor{tamred}{RGB}{228,0,43}
\definecolor{tamgrey}{RGB}{112,115,115}

\setbeamercolor{title}{fg = maroon}
\setbeamercolor{frametitle}{fg = tamwhite, bg = maroon}
\setbeamercolor{structure}{fg = tamgrey, bg = tamyellow}

\hypersetup{
    colorlinks=true,
    linkcolor=blue,
    filecolor=magenta,      
    urlcolor=cyan,
}

% Note, to include a LaTeX object (like output table), use:
% \input{file_name.tex}
% To include a graph or image, use:
% \includegraphics[scale=0.5]{file_name.png}

\title{Personal Finance and Forecasting}
%\author{Michael Black\inst{1}}
%\institute[]{
 %   \inst{1}%
  %  Department of Agricultural Economics\\
   % Texas A\&M University
%}
\date{AGEC 317: Economic Analysis for Agribusiness Management \\ Instructor: Michael Black}

\titlegraphic{\begin{flushright} \vspace{6.5cm} \includegraphics[width=1.5cm]{agec.png} \end{flushright}}


%%%%%%%%%%%%%%%%%%%%%%%%%%%%%%%%%%%%%%%%%%%%%%%%%%%%%%%%%%%%%%%%%%%%%
%%%%%%%%%%%%%%%%%%%%%%%%%%%%%%%%%%%%%%%%%%%%%%%%%%%%%%%%%%%%%%%%%%%%%
\begin{document}

%%%%%%%%%%%%%%%%%%%%%%%%%%%%%%%%%%%%%%%%%%%%%%%%%%%%%%%%%%%%%%%%%%%%%
\begin{frame}
  \titlepage
\end{frame}
%%%%%%%%%%%%%%%%%%%%%%%%%%%%%%%%%%%%%%%%%%%%%%%%%%%%%%%%%%%%%%%%%%%%%
\begin{frame}{An amuse-bouche}
Moving beyond supply and demand analysis, we can do some cool things with econometrics that can benefit us personally. This lecture is about the application of econometrics to building a portfolio of investments, and forecasting future behavior. But we are looking at the tip of the iceberg! For more, take a class dedicated to financial analysis. 
\end{frame}
%%%%%%%%%%%%%%%%%%%%%%%%%%%%%%%%%%%%%%%%%%%%%%%%%%%%%%%%%%%%%%%%%%%%%
\begin{frame}{Time Series}
Time series data had many observations over time for a single entity. 
\begin{center}
	\includegraphics[scale=0.5]{ts.png}
\end{center}
\end{frame}
%%%%%%%%%%%%%%%%%%%%%%%%%%%%%%%%%%%%%%%%%%%%%%%%%%%%%%%%%%%%%%%%%%%%%
\begin{frame}{Time Series}
\begin{itemize}
	\item Stock prices
	\item Crude oil
	\item Cattle prices
	\item Crime rates
	\item Housing expenditures
	\item etc
\end{itemize}
\end{frame}
%%%%%%%%%%%%%%%%%%%%%%%%%%%%%%%%%%%%%%%%%%%%%%%%%%%%%%%%%%%%%%%%%%%%%
\begin{frame}{Investing}
Let's say you want to invest in a or some stocks. You may be interested in:
\begin{itemize}
	\item \textbf{Expected return}: what you expect an investment will return
	\item \textbf{Risk}: the likelihood that the real return will differ from the expected return
\end{itemize}
\end{frame}
%%%%%%%%%%%%%%%%%%%%%%%%%%%%%%%%%%%%%%%%%%%%%%%%%%%%%%%%%%%%%%%%%%%%%
\begin{frame}{Investing}
Return for a time-series of stock prices:
$$Return_t = \frac{(Price_t - Price_{t-1})}{Price_{t-1}}$$
\end{frame}
%%%%%%%%%%%%%%%%%%%%%%%%%%%%%%%%%%%%%%%%%%%%%%%%%%%%%%%%%%%%%%%%%%%%%
\begin{frame}{Investing}
Expected return for a time-series of stock prices:
$$ER = \frac{\sum_{t-T}^t Return_t}{T}$$
...or the average of returns over a time period with $T$ increments.
\end{frame}
%%%%%%%%%%%%%%%%%%%%%%%%%%%%%%%%%%%%%%%%%%%%%%%%%%%%%%%%%%%%%%%%%%%%%
\begin{frame}{Investing}
\emph{Risk} is the likelihood that the real return is different than expected return. Two broad types of risk:
\begin{itemize}
	\item Systematic risk: risk that affects the entire system
	\item Unsystematic risk: risk that affects an individual stock (eg company)
\end{itemize}
\end{frame}
%%%%%%%%%%%%%%%%%%%%%%%%%%%%%%%%%%%%%%%%%%%%%%%%%%%%%%%%%%%%%%%%%%%%%
\begin{frame}{Investing}
It is hard to protect yourself from systematic risk: COVID-19 is affecting everyone, so how could you have adjusted your investment strategy to protect against this? \\
We will focus on protecting ourselves against unsystematic risk, through \emph{diversification}.
\end{frame}
%%%%%%%%%%%%%%%%%%%%%%%%%%%%%%%%%%%%%%%%%%%%%%%%%%%%%%%%%%%%%%%%%%%%%
\begin{frame}{Investing}
Two broad ways of measuring the risk of investing in a stock:
\begin{itemize}
	\item \textbf{Standard deviation}: measures the volatility of the stock by looking at the stocks historical returns
	\item \textbf{Beta}: measures the volatility of the stock by comparing returns of a stock to a standard market benchmark, like the S\&P 500.
\end{itemize}
\end{frame}
%%%%%%%%%%%%%%%%%%%%%%%%%%%%%%%%%%%%%%%%%%%%%%%%%%%%%%%%%%%%%%%%%%%%%
\begin{frame}{Investing}
Standard deviation is a formula we have already seen:
$$SD = \sqrt{\frac{\sum_{i=1}^n(x_i-\bar{x})^2}{n-1}}$$
If the standard deviation is high, the stock is volatile. This is considered high risk, which is attractive for aggressive-growth investment strategies, and unattractive for conservative-growth strategies.
\end{frame}
%%%%%%%%%%%%%%%%%%%%%%%%%%%%%%%%%%%%%%%%%%%%%%%%%%%%%%%%%%%%%%%%%%%%%
\begin{frame}{Investing}
Beta is the measurement of risk of a stock compared to a market benchmark:
$$\beta = \frac{COV(R_s, R_m)}{VAR(R_m)}$$
where $R_s$ is the return of an individual stock, and $R_m$ is the return of the market benchmark. 
\end{frame}
%%%%%%%%%%%%%%%%%%%%%%%%%%%%%%%%%%%%%%%%%%%%%%%%%%%%%%%%%%%%%%%%%%%%%
\begin{frame}{Investing}
Hold the phone! Does that definition of $\beta$ look familiar? That's the \emph{exact} same definition we derived for the estimate of a slope coefficient in a linear regression! Clearly, we'll be able to use a linear regression here...
\end{frame}
%%%%%%%%%%%%%%%%%%%%%%%%%%%%%%%%%%%%%%%%%%%%%%%%%%%%%%%%%%%%%%%%%%%%%
\begin{frame}{Investing}
Importantly, we can diversify risk by creating a \emph{portfolio} of investments. The portfolio is a collection of stocks we invest in simultaneously, and that portfolio can be designed to minimize the amount of unsystematic risk we are exposed to.
\end{frame}
%%%%%%%%%%%%%%%%%%%%%%%%%%%%%%%%%%%%%%%%%%%%%%%%%%%%%%%%%%%%%%%%%%%%%
\begin{frame}{Investing}
The standard deviation of a portfolio:
$$\sigma_p = \sqrt{w^2_1\sigma^2_1 + w^2_2\sigma^2_2 + 2w_1w_2\sigma_1\sigma_2\rho_{1,2}}$$
where $w_i$ is the weight (in percentage terms) of a stock in the portfolio, $\sigma^2_i$ is the variance of returns of stock $i$, and $\rho_{1,2}$ is the correlation between stock 1 and 2. As you get more stocks in the portfolio, this definition becomes \emph{long}. We won't do that here. \\
The beta of a portfolio:
$$\beta = \frac{COV(R_p, R_m)}{VAR(R_m)}$$
where $R_p$ is the return of the portfolio, and $R_m$ is the return of the market.
\end{frame}
%%%%%%%%%%%%%%%%%%%%%%%%%%%%%%%%%%%%%%%%%%%%%%%%%%%%%%%%%%%%%%%%%%%%%
\begin{frame}{Investing}
Let's pretend to invest some initial principle in a single stock and a portfolio of two stocks, and decide how to diversify...
\end{frame}
%%%%%%%%%%%%%%%%%%%%%%%%%%%%%%%%%%%%%%%%%%%%%%%%%%%%%%%%%%%%%%%%%%%%%
\begin{frame}{Forecasting}
So we can look at the past to understand current conditions. What about trying to predict the future? This is known as \textbf{forecasting}.
\end{frame}
%%%%%%%%%%%%%%%%%%%%%%%%%%%%%%%%%%%%%%%%%%%%%%%%%%%%%%%%%%%%%%%%%%%%%
\begin{frame}{Forecasting}
Some preliminaries:
\begin{itemize}
	\item We aren't going to incorporate uncertainty, so these forecasting models aren't as good as they could be.
	\item ``A model is always wrong, but a good model is useful"
\end{itemize}
\end{frame}
%%%%%%%%%%%%%%%%%%%%%%%%%%%%%%%%%%%%%%%%%%%%%%%%%%%%%%%%%%%%%%%%%%%%%
\begin{frame}{Forecasting}
We will focus on:
\begin{itemize}
	\item Static models
	\item Finite discrete lagged models (FDL)
\end{itemize}
\end{frame}
%%%%%%%%%%%%%%%%%%%%%%%%%%%%%%%%%%%%%%%%%%%%%%%%%%%%%%%%%%%%%%%%%%%%%
\begin{frame}{Static model}
A static model is:
$$y_t = \beta_0 + \beta_1x_t + u_t$$
where we say that $\beta_1$ is the \emph{contemporaneous} effect of x on y. 
\end{frame}
%%%%%%%%%%%%%%%%%%%%%%%%%%%%%%%%%%%%%%%%%%%%%%%%%%%%%%%%%%%%%%%%%%%%%
\begin{frame}{Static model: examples}
\begin{eqnarray}
	\nonumber inflation_t&=&\beta_0+\beta_1unemp_t + u_t \\
	\nonumber murder_t&=&\beta_0+\beta_1conviction_t + \beta_2unemp_t + u_t
\end{eqnarray}
\end{frame}
%%%%%%%%%%%%%%%%%%%%%%%%%%%%%%%%%%%%%%%%%%%%%%%%%%%%%%%%%%%%%%%%%%%%%
\begin{frame}{Finite discrete lag models}
In FDL models, we allow the independent variables to be lagged. The x of today \emph{and} yesterday are allowed to affect y today. In fact, we can lag x any number of times:
$$y_t=\alpha + \beta_0x_t + \beta_1x_{t-1} + \beta_2x_{t-2}+\cdots + \beta_qx_{t-q}$$
\end{frame}
%%%%%%%%%%%%%%%%%%%%%%%%%%%%%%%%%%%%%%%%%%%%%%%%%%%%%%%%%%%%%%%%%%%%%
\begin{frame}{Time trend}
Huge potential problem: is y changing because of x, or is it just naturally changing over time? We can get a better answer with a \textbf{time trend}.
\end{frame}
%%%%%%%%%%%%%%%%%%%%%%%%%%%%%%%%%%%%%%%%%%%%%%%%%%%%%%%%%%%%%%%%%%%%%
\begin{frame}{Forecasting}
With time series data, we can make predictions about some outcome in the future
\begin{itemize}
	\item Linear trend models
	\item Moving average (MA) models
	\item Autoregressive (AR) models
\end{itemize}
Before any model, you should plot the data as a line graph: $y_t$ on the y-axis, and time on the x-axis.
\end{frame}
%%%%%%%%%%%%%%%%%%%%%%%%%%%%%%%%%%%%%%%%%%%%%%%%%%%%%%%%%%%%%%%%%%%%%
\begin{frame}{Linear time trend}
$$y_t=\alpha + \beta_1t+\varepsilon_t$$
where $t=(1,\cdots, T)$. Then, $\beta_1$ measures the contemporaneous change in y given a one unit passage of time. We can add other explanatory variables:
$$y_t=\alpha + \beta_1t+\delta_0x_t + \delta_1x_{t-1}+\varepsilon_t$$
and $\beta_1$ measures the same thing, but now $\delta_0$ measures the contemporaneous effect of x on y, \emph{holding time constant}.
\end{frame}
%%%%%%%%%%%%%%%%%%%%%%%%%%%%%%%%%%%%%%%%%%%%%%%%%%%%%%%%%%%%%%%%%%%%%
\begin{frame}{Nonlinear time trend}
Exponential time trend:
$$\ln(y)_t=\alpha + \beta_1t+\varepsilon_t$$
Quadratic time trend:
$$y_t=\alpha + \beta_1t+\beta_2t^2+\varepsilon_t$$
\end{frame}
%%%%%%%%%%%%%%%%%%%%%%%%%%%%%%%%%%%%%%%%%%%%%%%%%%%%%%%%%%%%%%%%%%%%%
\begin{frame}{Example}
Consider these models on investment as a function of the housing price index. Without a time trend we get a \textbf{spurious relationship}.
\begin{center}
	\includegraphics[scale=0.5]{spur.png}
\end{center}
\end{frame}
%%%%%%%%%%%%%%%%%%%%%%%%%%%%%%%%%%%%%%%%%%%%%%%%%%%%%%%%%%%%%%%%%%%%%
\begin{frame}{Seasonality}
What if we think y depends on the season? Maybe demand for consumer goods spikes in November and December. We can capture these seasonal effects with dummy variables:
$$y_t = \alpha + \beta_1feb_t +\beta_2mar_t+\cdots+\beta_{11}dec_t+\cdots$$
where we exclude one base case.
\end{frame}
%%%%%%%%%%%%%%%%%%%%%%%%%%%%%%%%%%%%%%%%%%%%%%%%%%%%%%%%%%%%%%%%%%%%%
\begin{frame}{Seasonality}
\begin{center}
	\includegraphics[scale=0.5]{season.png}
\end{center}
\end{frame}
%%%%%%%%%%%%%%%%%%%%%%%%%%%%%%%%%%%%%%%%%%%%%%%%%%%%%%%%%%%%%%%%%%%%%
\begin{frame}{Time series regression assumptions}
\begin{itemize}
	\item Linear in parameters
	\item No perfect colinearity
	\item Zero conditional mean of error
	\item Homoskedasticity
	\item No serial correlation (error between time is uncorrelated)
	\item Normality of errors
\end{itemize}
\end{frame}
%%%%%%%%%%%%%%%%%%%%%%%%%%%%%%%%%%%%%%%%%%%%%%%%%%%%%%%%%%%%%%%%%%%%%
\begin{frame}{Time series regression assumptions}
Under the first three assumptions, OLS is unbiased. Under the first five assumptions, OLS estimators are BLUE. Under all six assumptions, we can make the same type of inferences from cross-sectional data on time series data. 
\end{frame}
%%%%%%%%%%%%%%%%%%%%%%%%%%%%%%%%%%%%%%%%%%%%%%%%%%%%%%%%%%%%%%%%%%%%%
\begin{frame}{Moving average}
With MA models, the outcome of interest in the future is predicted using average of the last $p$ observations:
$$\hat{y}_{t+1}=\frac{y_t+y_{t-1}+\cdots+y_{t-p+1}}{p}$$
\end{frame}
%%%%%%%%%%%%%%%%%%%%%%%%%%%%%%%%%%%%%%%%%%%%%%%%%%%%%%%%%%%%%%%%%%%%%
\begin{frame}{Autoregressive models}
With AR models, the outcome of interest in the future is predicted using a linear combination of the last $p$ observations:
$$\hat{y}_{t+1}=\alpha + \beta_0y_t+\beta_1y_{t-1}+\cdots+\beta_{t-p}y_{t-p+1}$$
With $p$ lags, the model above is known as an AR(p) model.
\end{frame}
%%%%%%%%%%%%%%%%%%%%%%%%%%%%%%%%%%%%%%%%%%%%%%%%%%%%%%%%%%%%%%%%%%%%%
\begin{frame}{How to forecast}
Linear time trend
\begin{enumerate}
	\item Recover coefficients on the time trend model
	\item Use coefficients to estimate any number of periods ahead
\end{enumerate}
\end{frame}
%%%%%%%%%%%%%%%%%%%%%%%%%%%%%%%%%%%%%%%%%%%%%%%%%%%%%%%%%%%%%%%%%%%%%
\begin{frame}{How to forecast}
Simple moving average:
\begin{enumerate}
	\item Decide on how many periods back to go
	\item Use the average of the previous periods to predict one step ahead
	\item Use the predicted step ahead plus other previous periods to predict the next step, and so on...
\end{enumerate}
\end{frame}
%%%%%%%%%%%%%%%%%%%%%%%%%%%%%%%%%%%%%%%%%%%%%%%%%%%%%%%%%%%%%%%%%%%%%
\begin{frame}{How to forecast}
AR(p) models
\begin{enumerate}
	\item Choose number of lags (p)
	\item Estimate model to recover $\hat{\beta_0}, \hat{\beta_1},\cdots,\hat{\beta_p}$
	\item Use the estimated coefficients to predict one step ahead, repeat
\end{enumerate}
\end{frame}
%%%%%%%%%%%%%%%%%%%%%%%%%%%%%%%%%%%%%%%%%%%%%%%%%%%%%%%%%%%%%%%%%%%%%
\begin{frame}{How to forecast}
What is a good forecast? Lots of measures, but we'll focus on \textbf{MSE}: mean-squared error. The error is:
$$\hat{\varepsilon}_{t+h} = y_{t+h} - \hat{y}_{t+h}$$
And the MSE is:
$$\frac{1}{N}\sum_{j=t}^{t+N-1}\hat{\varepsilon}^2_{j+h}$$
\end{frame}
%%%%%%%%%%%%%%%%%%%%%%%%%%%%%%%%%%%%%%%%%%%%%%%%%%%%%%%%%%%%%%%%%%%%%
\begin{frame}{Practice}
Let's go back to our stock prices and do some forecasting, and decide which model works best for our data.
\end{frame}
%%%%%%%%%%%%%%%%%%%%%%%%%%%%%%%%%%%%%%%%%%%%%%%%%%%%%%%%%%%%%%%%%%%%%
%%%%%%%%%%%%%%%%%%%%%%%%%%%%%%%%%%%%%%%%%%%%%%%%%%%%%%%%%%%%%%%%%%%%%
\end{document}